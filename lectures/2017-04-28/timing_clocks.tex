\documentclass[twoside]{article}
\setlength{\oddsidemargin}{0.25 in}
\setlength{\evensidemargin}{-0.25 in}
\setlength{\topmargin}{-0.6 in}
\setlength{\textwidth}{6.5 in}
\setlength{\textheight}{8.5 in}
\setlength{\headsep}{0.75 in}
\setlength{\parindent}{0 in}
\setlength{\parskip}{0.1 in}

\usepackage{graphicx}
\usepackage{url}

%
% The following commands sets up the lecnum (lecture number)
% counter and make various numbering schemes work relative
% to the lecture number.
%
\newcounter{lecnum}
\renewcommand{\thepage}{\thelecnum-\arabic{page}}
\renewcommand{\thesection}{\thelecnum.\arabic{section}}
\renewcommand{\theequation}{\thelecnum.\arabic{equation}}
\renewcommand{\thefigure}{\thelecnum.\arabic{figure}}
\renewcommand{\thetable}{\thelecnum.\arabic{table}}
\newcommand{\dnl}{\mbox{}\par}

%
% The following macro is used to generate the header.
%
\newcommand{\lecture}[4]{
  \pagestyle{myheadings}
  \thispagestyle{plain}
  \newpage
  \setcounter{lecnum}{#1}
  \setcounter{page}{1}
  \noindent
  \begin{center}
  \framebox{
     \vbox{\vspace{2mm}
   \hbox to 6.28in { {\bf COMPSCI~630~~~Systems
                       \hfill Spring 2017} }
      \vspace{4mm}
      \hbox to 6.28in { {\Large \hfill Lecture #1: #2  \hfill} }
      \vspace{2mm}
      \hbox to 6.28in { {\it Lecturer: #3 \hfill Scribe(s): #4} }
     \vspace{2mm}}
  }
  \end{center}
  \markboth{Lecture {#1}: #2}{Lecture {#1}: #2}
  \vspace*{4mm}
}

%
% Convention for citations is authors' initials followed by the year.
% For example, to cite a paper by Leighton and Maggs you would type
% \cite{LM89}, and to cite a paper by Strassen you would type \cite{S69}.
% (To avoid bibliography problems, for now we redefine the \cite command.)
%
\renewcommand{\cite}[1]{[#1]}

% \input{epsf}

%Use this command for a figure; it puts a figure in wherever you want it.
%usage: \fig{NUMBER}{FIGURE-SIZE}{CAPTION}{FILENAME}
\newcommand{\fig}[4]{
           \vspace{0.2 in}
           \setlength{\epsfxsize}{#2}
           \centerline{\epsfbox{#4}}
           \begin{center}
           Figure \thelecnum.#1:~#3
           \end{center}
   }

% Use these for theorems, lemmas, proofs, etc.
\newtheorem{theorem}{Theorem}[lecnum]
\newtheorem{lemma}[theorem]{Lemma}
\newtheorem{proposition}[theorem]{Proposition}
\newtheorem{claim}[theorem]{Claim}
\newtheorem{corollary}[theorem]{Corollary}
\newtheorem{definition}[theorem]{Definition}
\newenvironment{proof}{{\bf Proof:}}{\hfill\rule{2mm}{2mm}}

% Some useful equation alignment commands, borrowed from TeX
\makeatletter
\def\eqalign#1{\,\vcenter{\openup\jot\m@th
 \ialign{\strut\hfil$\displaystyle{##}$&$\displaystyle{{}##}$\hfil
     \crcr#1\crcr}}\,}
\def\eqalignno#1{\displ@y \tabskip\@centering
 \halign to\displaywidth{\hfil$\displaystyle{##}$\tabskip\z@skip
   &$\displaystyle{{}##}$\hfil\tabskip\@centering
   &\llap{$##$}\tabskip\z@skip\crcr
   #1\crcr}}
\def\leqalignno#1{\displ@y \tabskip\@centering
 \halign to\displaywidth{\hfil$\displaystyle{##}$\tabskip\z@skip
   &$\displaystyle{{}##}$\hfil\tabskip\@centering
   &\kern-\displaywidth\rlap{$##$}\tabskip\displaywidth\crcr
   #1\crcr}}
\makeatother

% **** IF YOU WANT TO DEFINE ADDITIONAL MACROS FOR YOURSELF, PUT THEM HERE:



% Some general latex examples and examples making use of the
% macros follow.

\begin{document}

%FILL IN THE RIGHT INFO.
%\lecture{**LECTURE-NUMBER**}{**DATE**}{**LECTURER**}{**SCRIBE**}
\lecture{13}{April 28}{Emery Berger}{Manish Motwani, Sam Baxter}

\section{Times, Clocks, and Stuff Like That}

One of the major issues in distributed systems is dealing with time because the
assumption (made in everyday activites) that time is universal doesn't hold
true. This is because latency in communication can cause non-deterministic
ordering of events and clocks to fall out of sync.

If we could attach a \emph{perfect} timestamp to each event occurence, we could
obtain a \emph{total order} on those events. Unfortunately, perfect timestamps
are infeasible in distributed systems because the latency of message passing
communication and clock drift forces the loss of synchronization.

\subsection{How are clocks synchronized?}
\begin{enumerate}
    \item Atomic Clocks - The most accurate time measuring device. Uses the
        frequency of transmission of atoms under some light in the
        electromagnetic spectrum as a frequency standard for timekeeping. These
        are expensive (costs \$25000) and are not practical for large
        or personal deployments.
    \item Quartz Crystal - Quartz is known to vibrate at a certain frequency
        under some electronic oscillator. This creates a precise signal that
        produces a much more accurate clock than digital clocks, but less
        accurate than atomic clocks. The synchronization problem still persists,
        and several quartz clocks can drift further and further apart.
\end{enumerate}
The time delay in synchronizing these clocks is nondeterministic. The
\emph{Network Time Protocol} (NTP) is a common standard for synchronizing the
clocks of moder distributed servers. However, this synchronization is
coarse-grained (i.e. clocks may be guaranteed to be within a day, hours,
minutes, or seconds of sync with one another, but not fine-grained enough to
guarantee against inversions in the perceived order of events).

Lamport described a way to determine a global ordering of events in a
distributed system by defining a "happens-before" relationship between two
events, establishing a \emph{total order}. Lamport used \emph{logical clocks}
(a.k.a. Lamport clocks), which are essentially global counters (as opposed to
perfect timestamps) that increment each event and are transmitted to all parts
of a distributed system in every communication.

\subsection{Vector Clocks}
Vector clocks generalize Lamport clocks by maintaining essentially a vector of
logical clocks, one for each thread or process in the distributed system. A
vector clock must be maintained for \emph{each} variable in the program. They
encode the happens-before relation on all variable accesses/modifications,
establishing a total order on these events.

Consider the following example.

\subsection{Race Detection}
The happens-before relation can be used to detect potential race conditions in
variable accesses and modifications. Vector clocks encode this relationship, so
they lend to a \emph{precise} notion of race-detection (i.e. if a race were to
occur, it would be detected, and any races it detects are legitimate, potential
races).

With the precision of vector clocks comes performance overhead. They require
$O(n)$ space (linear in the number of threads/processes) per variable and vector
clock operations (comparisons to determine ordering) are $O(n)$ time as as they
require a linear traversal of the vectors.

\subsection{Alternative Approaches to Race Detection}
To address the performance overhead of vector clocks, some techniques sacrifice
precision. ERASER is a Java race detector that uses such an approach, employing
a \emph{LockSet} analysis algorithm to detect potential races (potentially
giving false-positives). The key insight here is that if multiple processes work
on a shared variable without holding a lock, then there is a potential race.
Note that if programs use alternative means of establishing ordering (rather
than locking primitives), LockSet cannot detect that they may remain mutually
exclusive.

\subsection{FastTrack - Prof. Stephen Freund Guest Lecturer}
FastTrack gives the precision of vector clocks without the space/time overhead
required by that technique. Freund exploited the realization that in the
majority of scenarios, we only need to remember the \emph{very last write} to a
shared variable instead of a vector clock encoding its entire history. As long
as a race has yet to be detected, there is still a total order on writes, and so
FastTrack replaces linear-sized vector clocks with an \emph{epoch} encoding the
thread id and counter of the last write to a variable. This preserves the
happens-before relation before any races have occured, so FastTrack guarantees
that it will precisely detect at least the first race to occur on a variable.



\end{document}
