\documentclass[twoside]{article}
\setlength{\oddsidemargin}{0.25 in}
\setlength{\evensidemargin}{-0.25 in}
\setlength{\topmargin}{-0.6 in}
\setlength{\textwidth}{6.5 in}
\setlength{\textheight}{8.5 in}
\setlength{\headsep}{0.75 in}
\setlength{\parindent}{0 in}
\setlength{\parskip}{0.1 in}

\usepackage{graphicx}
\usepackage{url}
\usepackage{amsmath}
%\usepackage{chngcntr}
%\counterwithin{figure}{subsection}

%
% The following commands sets up the lecnum (lecture number)
% counter and make various numbering schemes work relative
% to the lecture number.
%
\newcounter{lecnum}
\renewcommand{\thepage}{\thelecnum-\arabic{page}}
\renewcommand{\thesection}{\thelecnum.\arabic{section}}
\renewcommand{\thesubsection}{\thesection.\arabic{subsection}}
\renewcommand{\theequation}{\thesubsection.\arabic{equation}}
\renewcommand{\thefigure}{\thesubsection.\arabic{figure}}
\renewcommand{\thetable}{\thelecnum.\arabic{table}}
\newcommand{\dnl}{\mbox{}\par}

%
% The following macro is used to generate the header.
%
\newcommand{\lecture}[4]{
  \pagestyle{myheadings}
  \thispagestyle{plain}
  \newpage
  \setcounter{lecnum}{#1}
  \setcounter{page}{1}
  \noindent
  \begin{center}
  \framebox{
     \vbox{\vspace{2mm}
   \hbox to 6.28in { {\bf CMPSCI~630~~~Systems
                       \hfill Spring 2017} }
      \vspace{4mm}
      \hbox to 6.28in { {\Large \hfill Lecture #1  \hfill} }
%       \hbox to 6.28in { {\Large \hfill Lecture #1: #2  \hfill} }
      \vspace{2mm}
      \hbox to 6.28in { {\it Lecturer: #3 \hfill Scribe: #4} }
     \vspace{2mm}}
  }
  \end{center}
  \markboth{Lecture #1: #2}{Lecture #1: #2}
  \vspace*{4mm}
}

%
% Convention for citations is authors' initials followed by the year.
% For example, to cite a paper by Leighton and Maggs you would type
% \cite{LM89}, and to cite a paper by Strassen you would type \cite{S69}.
% (To avoid bibliography problems, for now we redefine the \cite command.)
%
\renewcommand{\cite}[1]{[#1]}

% \input{epsf}

%Use this command for a figure; it puts a figure in wherever you want it.
%usage: \fig{NUMBER}{FIGURE-SIZE}{CAPTION}{FILENAME}
\newcommand{\fig}[4]{
           \vspace{0.2 in}
           \setlength{\epsfxsize}{#2}
           \centerline{\epsfbox{#4}}
           \begin{center}
           Figure \thelecnum.#1:~#3
           \end{center}
   }

% Use these for theorems, lemmas, proofs, etc.
\newtheorem{theorem}{Theorem}[lecnum]
\newtheorem{lemma}[theorem]{Lemma}
\newtheorem{proposition}[theorem]{Proposition}
\newtheorem{claim}[theorem]{Claim}
\newtheorem{corollary}[theorem]{Corollary}
\newtheorem{definition}[theorem]{Definition}
\newenvironment{proof}{{\bf Proof:}}{\hfill\rule{2mm}{2mm}}

% Some useful equation alignment commands, borrowed from TeX
\makeatletter
\def\eqalign#1{\,\vcenter{\openup\jot\m@th
 \ialign{\strut\hfil$\displaystyle{##}$&$\displaystyle{{}##}$\hfil
     \crcr#1\crcr}}\,}
\def\eqalignno#1{\displ@y \tabskip\@centering
 \halign to\displaywidth{\hfil$\displaystyle{##}$\tabskip\z@skip
   &$\displaystyle{{}##}$\hfil\tabskip\@centering
   &\llap{$##$}\tabskip\z@skip\crcr
   #1\crcr}}
\def\leqalignno#1{\displ@y \tabskip\@centering
 \halign to\displaywidth{\hfil$\displaystyle{##}$\tabskip\z@skip
   &$\displaystyle{{}##}$\hfil\tabskip\@centering
   &\kern-\displaywidth\rlap{$##$}\tabskip\displaywidth\crcr
   #1\crcr}}
\makeatother

% **** IF YOU WANT TO DEFINE ADDITIONAL MACROS FOR YOURSELF, PUT THEM HERE:

% Some general latex examples and examples making use of the
% macros follow.

\begin{document}

%FILL IN THE RIGHT INFO.
%\lecture{**LECTURE-NUMBER**}{**DATE**}{**LECTURER**}{**SCRIBE**}
\lecture{18}{April 18}{Emery Berger}{Andrew Danise, Arjun Sreedharan}

\section{Databases}

\subsection{History}

Early Databases were very primitive and did not have simple querying capabilities.
CODASYL exposes an API for general purpose programming languages.
Lookups were done using a minimal abstraction built using iterators via a cursor.
It maintained an index made up of a bunch of pointers to data records.

The development of SQL (Structured Query Language) started out at IBM.
It was preceded by a language named Quel which was based on tuples.
IBM's Database research in the early stages followed 2 approaches
to building querying capability:

\begin{description}
  \item[$\bullet$ SQL]
  \item[$\bullet$ QBE] (Query By Example): QBE is a visual way of expressing queries.
      The graphical queries would look similar to ASCII art. QBE is very easy for
        untrained users to understand, but it is difficult to express complex queries.
\end{description}

\subsection{Structure}

Databases use B-Tree or B\texttt{+} Tree data structures. B-tree is a
generalization of a Binary Search Tree where each node can have multiple children.
B-trees are used to implement database indices. They maximize locality when
searching for a particular key. B-trees are very efficient when represented
on disk. Size of a B-tree node is typically a 4K page.

Unlike B-tree, B\texttt{+} trees don't have data associated with interior nodes and more keys fit on a page of memory.

\subsection{Partitioning}

\subsubsection{Range Partitioning}
In Range partitioning, data is partitions based on the range of
values of a particular key. For example, if \textit{Name}
is the partitioning key, all entries will with name
beginning in letter from A to J can be in one partition,
those beginning with K to P in another and the rest in
the third. This gives a good speedup if the queries can
be parallely executed in the different partitions.
The size of partitions may not be equal.

\subsection{Hash Partition}
Hash Partitioning is based on a hashing algorithm
that is applied to the partitioning key. The hashing
function evenly spreads the entries among partitions,
giving partitions approximately the same size. It is
ideal for balancing data storage evenly across machines or replicas.

There are also other methods for partitioning
like Horizontal and Vertical partitioning (sharding).
In Horizontal partitioning, entries in the database are
divided into different partitions. In Vertical Partitioning, the
fields of the entries are divided into different partitions.

\section{Query Optimization}

Sometimes the order in which the query is executed
has a drastic impact on the performance of the query.
In addition, sometimes the database can optimize queries
because it knows about some inherent properties of the data.
For example, if you run the the SQL query "select * from DB where salary > 10000 and name == "John"
at a company where everyone makes more than 10000 it will need to lookup all of the data. However,
if the database reorganized the query to first check if the name == John than the results would be
more limited. In order to perform such optimizations, databases store statistics on the type
of data they hold. They use this data to reorganize queries to achieve the best possible performance.
This is called Selectivity.

\section{MapReduce}

Databases are great for storing structured data. But they
do not scale well when unstructured on semi-structured data
is stored. Assume that an organization has to store many HTML
or XML documents. The overhead of query optimizer and other
management overhead for databases makes its scalability very
limited. This is where a restricted model of distributed
computing named MapReduce comes into the picture. MapReduce
is highly scalable with unstructured documents. MapReduce
works on the basis of \textit{map} and \textit{reduce} methods
provided by the user. It provides in-built fault tolerance,
and at every stage writes data to disk.

\end{document}
